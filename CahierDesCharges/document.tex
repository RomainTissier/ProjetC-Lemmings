\documentclass[a4paper]{article}
\usepackage[utf8]{inputenc}
\usepackage[T1]{fontenc}
\usepackage[francais]{babel}
\usepackage{fullpage}
\title{Cahier des charges fonctionnels}
\author{Marc Delpech, Yoni Levy, Matthieu Rousselle, Romain Tissier}
\date{2015}
\begin{document}
	\maketitle
	\newpage
	\tableofcontents
	\newpage
	\part{Présentation du jeu}
		\section{Orgine}
			Le jeux des Lemmings est un jeux vidéo de réflexion datant de 1991. Après avoir joué à sa version DHTML, nous avons décider de réaliser un clone de ce jeux pour le projet de C. 
		\section{But du jeux}
			Les lemmings sont des petits rongeur qui vivent en groupe. Dans le jeux ces rongeeurs apparraissent sur l'écran par une porte, évolue dans un environnement prédéterminé. Le joueur a pour mission de guider un maximum de Lemmings à la sortie. Pour cela, il peut affecter des actions à un lemmings: 
			\begin{itemize}
				\item "GRIMPER" : le lemmings peut alors escalader un obstacle vertical.
				\item "Voler" : le lemmings peut voler pour limiter les dégats d'une chute verticale. 
				\item "Bloquer" : le lemmings bloque le passage, les autres lemmings doivent faire demi-tour
   				\item "Exploser": le lemmings se sacrifie et explose après 5 secondes de compte à rebourd. Il réalise alors des dégats sur l'environnement extérieur et sur les lemmings. 
				\item "Construire": le lemmings pose des briques(12) pour former un escalier. 
    				\item "Pelleter": le lemmings creuse une paroi horizontalement
				\item "Creuser": le lemmings creuse le sol verticalement
				\item "Miner": le lemmings creuse le sol horizontalement
			\end{itemize}
			Selon les niveaux, certaines actions sont autorisées(ou non) et/ou limités. Par exemple, on ne pourra faire grimper que 3 lemmings.  
		\section{Captures d'écrans}
	\part{Expression du besoin}
		un escalier est destructible
		l'entrée et la sortie ne sont pas destructibles
	\part{Décomposition du travail}
		\section{Création du menu}	
	\part{Charte graphique}
\end{document}
