\documentclass[a4paper]{article}
\usepackage[utf8]{inputenc}
\usepackage[T1]{fontenc}
\usepackage[francais]{babel}
\usepackage{fullpage}
\title{Cours de droit}
\author{Romain Tissier}
\date{2015}
\begin{document}
	\maketitle
	\newpage
	\tableofcontents
	\newpage
	\part{Introduction}
		\section{Qu'est ce que le droit?}
			Instinctivement nous faisons tous déjà du droit. Par exemple, nous distinguons la différence entre une règle juridique et une régle de conduite en société. Parmi ces règles juridique, nous en connaissons beaucoup comme le salaire minimal ou encore l'interdiction de rouler à contre-sens. Nous connaissons aussi la célèbre citation: \textit{"qui tacet consentire videtur",} du pape Boniface VIII (1235-1303) qui se traduit en français pas "qui ne dit mot consent".
		\section{Définition}
			Le droit est l'ensemble des règles juridiques qui régissent la vie des hommes en société; sanctionné le cas échéant par une contrainte exercée par l'autorité public. Mais attention, toute régle de vie en société n'est pas juridique. 
		\section{Distinction entre droit, justice, équité}
			\begin{quote}
				\textit{"Dieu nous garde de l'équité des parlements"}
			\end{quote}
			Afin de préserver la sécurité juridique, la loi impose au juge français de statuer en Droit et non en équité ou en fonction de sa propre idée du juste. 
		\section{Distinction entre droit et morale}
			La morale irrigue l'ensemble du droit mais ces deux notions sont différentes: 
			\begin{itemize}
				\item la morale vise la perfection de l'homme, le droit a seulement pour objectif le fonctionnement harmonieux de la société
				\item le domaine de la morale dépend de chaque individu et évolue avec le temps. Le droit se veut la plus large morale commune possible avec une certaine stabilité afin d'assurer la sécurité juridique. 
				\item la sanction morale est intérieur, elle est issue de la conscience. La sanction du droit est elle assurée de l'extérieur par l'autorité publique. 
			\end{itemize}
		\section{Distinction entre Droit et droit}
			Le Droit avec un grand "D" est le droit objectif, c'est l'ensemble des règles juridiques d'un pays donné. Il est différent du droit, qui lui est subjectif. Il correspond a une prérogative individuelle qui existe sur la tête d'une personne titulaire d'un droit. Au final, le Droit objectif confère des droits subjectifs aux acteurs de la vie juridique. 
	\newpage\part{La règle de droit}
		\section{Obligation et coercition}
			La régle de droit est obligatoire et coercitive\footnote{La coercition est l'action de contraindre, exercée sur quelqu'un, pour le forcer à agir ou à s'en abstenir.}. C'est un commandement et la régle peut être appliquée par contrainte. Par exemple, l'article 544 du code civil dispose\footnote{On dit toujours d'une loi qu'elle dispose} que: 
			\begin{quote}
				\textit{"La propriété d'un bien est disposer de la chose de la manière la plus absolue du moment qu'il n'en fait pas un usage prohibé par la loi"}
			\end{quote}
			Les sanctions peuvent être civiles (domages et intéret, éxécution forcée,...) mais parfois pénale(amende, peine emprisonnement voir réclusion). Attention, l'emprisonnement fait référence à un délit alors que la réclusion fait référence à un crime. 
		\section{Généralisation et abstraction}
			La règle de droit a vocation d'être appliquée sur tout le territoire francais et sur toute personne qui forme le corps social. La loi ne désigne jamais la personne nomément mais elle peut viser spécifiquement une catéforie de personnes. Elle est formulée en termes généraux et ne s'applique qu'aux cas particulier qu'à posteriori.
	\newpage\part{Les branches du droit}
		Le droit est un ensemble formé de sous ensemble. Ainsi il y a une distinction entre droit privé, droit public, droit nationnal, droit internationnal...
		\section{Distinction entre droit privé et droit public }
			La loi différe tout d'abord au niveau des objectifs: le droit public à pour but l'interet général alors que le droit privé à pour but l'interet privé. Son caractère différe aussi: le droit public est impératif, le droit privé est lui beaucoup plus libéral. Enfin les sanction différent : dans le droit public, l'état a certain privilège alors que dans le droit privé, les particuliers sont à égalité. 
		\section{Les branches du droit public} 
			\begin{itemize}
				\item droit constitutionnel, qui régit les régles concernant l'organisation de l'état
				\item droit administratif, le droit administratif est constitué de l'ensemble des règles définissant les droits et les obligations de l'administration.
				\item finances publiques, Les finances publiques sont l'étude des règles et des opérations relatives aux deniers publics. 
				\item droit internationnal public, qui régit les rapport entre êtat, cependant aucun état n'est obligé de le respecter. 
			\end{itemize}
		\section{Les branches du droit privé}
			Le droit privé traite des rapports entre particulier enter eux et particulier et collectivités privé (entreprise association... Mais non administrative)
			\begin{itemize}
				\item droit commun, civil, établit par le code civil(1804), par exemple il établit notamment les règles dans des domaines comme celui de la famille (mariages, etc...).
				\item droit commercial, réservé aux commerçant et particulier qui effectuent un acte de commerce. Le premier code de commerce date de 1807.
				\item droit de la consommation, qui régit les règles entre les consommateurs et les professionnels
				\item droit à la propriété intellectuelle: Les droits de la propriété industrielle concernent les brevets d'invention, les marques, les dessins et modèles. 
				\item droit privé international, qui régit les règles entre particulier avec élément d'extranéité\footnote{caractère de ce qui est étranger}(Exemple: divorce entre un français et une irlandaise marié en Allemagne et résident au royaume uni: il y a conflit de loi entre les pays, le droit privé international vise donc à trouver une solution)
			\end{itemize}
		\section{Les branches du droit mixte}
			La frontière entre le droit public et le droit privé n'est pas étanche. On parle alors de droit mixte. 
%TODO étoffer les définitions
			\begin{itemize}
				\item droit de l'union européenne 
				\item droit de l'environnement
				\item droit médical
				\item droit social, qui inclu le droit du travail(ex : SMIC, condition de licenciement...)
				\item droit pénal
				\item droit fiscal
			\end{itemize}
	\newpage\part{Les sources du droit}
		Les sources du droit sont les modes de création du droit: comment le droit est-il créé à notre époque?
		\section{Les sources directes du droit}
			\subsection{Les sources écrites}
				\subsubsection{Les sources nationnales}
					Ces sources sont composés selon la hiérarchie suivante:
%TODO shéma de la pyramide
					\begin{itemize}
						\item La constitution: c'est elle qui constitue la république. Elle établie le droit public et le droit constitutionnel. Son rôle est de gérer les rapports entre les organes essentiels de l'état et du gouvernement. Nous sommes actuellement sous la cinquième république, ainsi la france possède par exemple un parlement composé de l'assemblé nationnal et du sénat. La constitution francaise reprend en préambule la déclaration des droits de l'homme et du citoyen de 1789. 
						\item La loi au sens strict: c'est la loi ordinaire, elle est limité par l'article 34 de la constitution et ne peut dépasser les limites de cet article sous peine d'être censurée par le conseil constitutionel. Elle concerne notamment les impôts, les procédures pénales etc...
						\item L'ordonnance : le gouvernement peut prendre par ordonnance des mesures qui relèvent normalement de la loi.
						Ordonnance est donc la catégorie intermédiaire entre la loi et les règlements.
%TODO vérifier définition réglement
						\item Le réglement (ou décret autonome), la ou la loi n'a pas de compétence, le règlement s'applique. Le gouvernement peut prendre un décret autonome, mais uniquement après avoir consulté le conseil d'état.
						\item Le décret d'application: c'est la modalité pratique d'une loi, par exemple la fixation du salaire minimum est un décret d'application
%TODO définition d'un arrété
						\item Arrêté 
							\subitem - Arrêté ministeriel : pris par un ministre
							\subitem - Arrété prefectoral : pris par un prefet
							\subitem - Arrété municipal : pris par un maire
					\end{itemize}
				
					La hiérarchie est respécté, par exemple, un maire ne peut pas déposer un arrêté interdisant le droit de grève car la constitution l'empéche. Enfin tout en bas de la pyramide, on peut ajouter les circulaires ministériel qui sont en quelquesorte le mode d'emploi d'une loi mais qui n'est pas valable devant un juge. 
				\subsubsection{Les sources internationnales}
					Ces sources sont composés des traités internationnaux hors de l'union européene. Le cas d'un litige implicant la fraude fiscale ou la double imposition utilise ces sources. L'exemple de deux de ces sources internationnales sont la convention de san Francisco en 1945 qui est à l'origine de l'ONU ou la déclaration de marakeche en 1994 qui est à l'origine de l'OMC. L'article 55 de la constitution francaise précise qu'un traité internationnal ratifié par le parlement a une valeur supérieure à la loi francaise même si celle-ci est contraire ou postérieure. 
					
					Ces sources sont aussi composés des traités communautaires, les plus célèbres sont les traités européens:
					\begin{itemize}
						\item 1951 : Traité de Paris (Forme avec le traité de Rome et l'acte unique européen le droit primaire de l'union européene)
						\item 1957 : Traité de Rome
						\item 1986 : Acte unique européen
						\item 1992 : traité de l'union européene
						\item 1997 : traité d'amsterdam
						\item 2007 : traité de Lisbone
					\end{itemize}
					
					Le droit dérivé de l'union européenne est composé de :
					\begin{itemize}
						\item Réglement: équivalent d'une loi nationnale et applicatble dans chaque état
						\item directives: destiné à tout ou certain état, définissant des objectifs obligatoire à atteindre
						\item décisions : régle obligatoire imposé a certain état
						\item recommendation et avis: conseils prodigué aux état, peut servir à l'interprétation du droit dans les états de l'union. 
					\end{itemize}
			\subsection{Des sources non écrites}			
				\begin{quote}
					- "En traversant la France, on change plus souvent de lois que de chevaux", Voltaire
				\end{quote}
				\subsubsection{La coutume}
					\begin{quote}
						- "Une fois n'est pas coutume"
					\end{quote}
					Une coutume est une pratique répétée et habituelle qui tend à se poser en régle de droit. Elle implique répétition et généralisation sur une longue période. Le comportement doit être percu comme obligatoire par l'opinion ou le groupe visé. Une coutume peut biensur disparaitre d'elle même ou être supprimé par une loi qui lui sera toujours supéreieure. On peut prendre comme exemple de coutume les mineurs de fond, qui à la saint Barbe avaient un jour de congé payé. 
				\subsubsection{L'usage}
%TODO compléter la définition
					L'usage est une pratique courante en droit commercial, en droit internationnal et en droit au travail. Par exemple, une prime versé depuis plus de 20 ans dans une entreprise doit continuée à être versée. 
		\section{Les sources indirectes du droit}
			\subsection{La jurisprudence}
				La jurisprudence est l'ensemble des décisions rendues par les juridiction francaise et en particulier à la cours de cassation et au conseil d'état. Contrairement aux pays anglosaxon, ou l'on parle de "common law" ou elle a un role principal, en france elle ne garde qu'un role secondaire. 
			\subsection{La doctrine}
%TODO limite connaissance magistrat : déni de ?
				Un magistrat est limité à ses connaissance. La doctrine est l'ensemble des travaux consacrés à l'étude du droit et de toutes les opinions émise sur le droit. Elle a un double role: à la fois prédagogique et prospectif. Elle permet par exemple de trouver des défaut dans une législation et d'émetre de meilleurs proposition. La doctrine est libre d'exprimée son avis et est libre de critiqué la loi, par ailleurs, une doctrine est rarement unanime et susite souvent des controverses. 
				
				L'avis d'une doctrine ne s'impose jamais à un juge. Mais elle peut être consulté pour la proposition d'une loi(Qui implique le parlement) ou pour le projet d'une loi (Qui implique le gouvernement). Ainsi, même si la doctrine na pas de valeur, elle peut influencer, ou constituer un guide pou une décision et a donc une importante force de persuasion. 
	\newpage\part{Application de la régle de droit}
		\section{Compétence juridictionnelle}
%TODO vérifier assez éléments orthographe tribunal prudhomme
			On parle ici de demandeur/demanderesse(l'attaquant) et de défendeur/défenderesse (celui qui est attaqué). Il existe deux grandes familles de juridictions : les juridictions administratives qui sont en charge des litiges qui opposent les particuliers à l'administration et les juridictions judiciaires.			
			\subsection{Compétence d'attribution}
%TODO def à revoir
				La compétence d'attribution est l'aptitude à connaitre d'une affaire déterminée principalement par la nature du litige: selon la matière, l'affaire ne sera pas géré au même tribunal: en effet le tribunal de grande instance ne gère pas les même affaire que les prud'homme).
			\subsection{Compétence territoriale}
%TODO def à revoir
				La compétence territoriale est l'aptitude à connaitre d'une affaire déterminée par des critères géographiques. Le domaine pénal fait exception, car le lieu de l'infraction est le lieu de juridiction, idem pour le tribunal administratif. 
			\subsection{Problème du délai de prescription}
				Si le mauvais tribunal est saisi, le temps de décision d'incompétence peut être supérieur au délai de prescription. Par exemple, une diffamation sur internet a une durée de prescription de 3 mois, le temps que l'on remarque la diffamation, si l'on se trompe de tribunal, on peut facilement dépasser les 3 mois. 
			\subsection{Notion de juridiction}
				Une juridiction est un organe crée par la loi qui a pour but principal de trancher les littiges et de dire la loi. Ainsi tout les tribunaux et toutes les cours sont des juridictions.
				\subsubsection{Les juridictions d'ordre administratives}	
					Les juridictions d'ordre administratives connaissent des litiges qui opposent les particulier et les administrations. À la suite d'un passage au tribunal administratif (il n'en existe que 42) : on peut faire le choix d'interjeter\footnote{On interjecte appel, mais attention on ne "fait" jamais appel} appel, c'est-à-dire former un recours devant la Cour administrative d'appel (il n'en existe que 8) contre la décision prise par le Tribunal administratif. Ces Cours permettent de désengorger le Conseil d'État qui peut toujours être mis à contribution en cas de pourvoi en cassation (=contestation de la décision de la Cour). Le Conseil d'État est composé d'une formation administrative et d'une formation contentieuse, qui sont divisées en sous-sections afin de répondre au mieux aux différentes sollicitations.	
%TODO La phrase précédente est incompléte
%TODO faire lien pdf gouvernement : on ne fait pas appel on interjecte 
%TODO http://www.justice.gouv.fr/organisation-de-la-justice-10031/lordre-judiciaire-10033/
			\subsection{Point sur les magistrats}
				Un magistrat peut être membre du parquet ou juge du siège:
				\begin{itemize}
					\item Les magistrats du parquet conduisent et jugent les affaires (ils sont indépendants et autonomes)
					\item Les juges du siège défendent l'État et sont sous l'autorité du Garde des Sceaux.
				\end{itemize}
\end{document}